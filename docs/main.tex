\documentclass[a4paper]{article}

% =========================
% Encoding & language
% =========================
\usepackage[utf8]{inputenc}
\usepackage{vntex}
\usepackage[english,vietnam]{babel}

% Đổi caption sang tiếng Anh cho cả tiếng Việt
\addto\captionsvietnamese{%
  \renewcommand{\contentsname}{Table of Contents}%
  \renewcommand{\listfigurename}{List of Figures}%
  \renewcommand{\listtablename}{List of Tables}%
  \renewcommand{\figurename}{Figure}%
  \renewcommand{\tablename}{Table}%
}

% =========================
% Basic packages
% =========================
\usepackage{amsmath,amssymb,amsfonts,amsthm}
\usepackage{array,booktabs,multirow,longtable,tabularx}
\usepackage{graphicx}
\usepackage[dvipsnames]{xcolor}
\usepackage[most]{tcolorbox}
% =========================
% Pretty box (tcolorbox) for logs / testcases
% =========================
\newtcolorbox{expbox}[1][]{
  enhanced,
  breakable,
  colback=gray!5,
  colframe=blue!70!black,
  boxrule=0.8pt,
  arc=2mm,
  left=2mm,right=2mm,top=1mm,bottom=1mm,
  before skip=6pt,
  after skip=6pt,
  title=\textbf{Expected log},
  fonttitle=\small\bfseries,
  coltitle=black,
  #1
}

\usepackage{enumerate}
\usepackage{multicol}
\usepackage{rotating}
\usepackage{geometry}
\usepackage{setspace}
\usepackage{lastpage}
\usepackage{float}      % [H] cho float đứng yên
\usepackage{placeins}   % \FloatBarrier để chốt float
\usepackage{tikz}

%=========================================================
% Colors + TikZ styles for pipeline timeline
% =========================================================
\definecolor{IDc}{RGB}{255,249,196}
\definecolor{EXc}{RGB}{255,224,178}
\definecolor{MEMc}{RGB}{187,222,251}
\definecolor{WBc}{RGB}{179,229,252}

% nhớ có patterns + arrows.meta
\usetikzlibrary{shapes.geometric,positioning,arrows.meta,patterns}

\tikzset{
  stg/.style={
    draw,
    trapezium,
    trapezium left angle=78,
    trapezium right angle=78,
    trapezium stretches=true,
    minimum height=5.5mm,
    minimum width=11mm,
    align=center,
    text centered,
    text height=1.6ex,
    text depth=0.25ex,
    inner xsep=0pt,
    inner ysep=0pt,
    font=\scriptsize,
    line join=round
  },
  bubble/.style={
    stg,
    fill=gray!20,
    pattern=north east lines,
    pattern color=gray!70
  },
  fwdarrow/.style={->, line width=0.9pt, draw=red}
}


% =========================================================
% Listings style for RISC-V asm (override global lstset)
% =========================================================
\lstdefinestyle{asm}{
  language={},
  basicstyle=\ttfamily\small,
  columns=fullflexible,
  numbers=none,
  frame=single,
  breaklines=true,
  keepspaces=true,
  showstringspaces=false,
  backgroundcolor=\color{gray!10}
}


\usetikzlibrary{arrows,snakes,backgrounds}
\newcommand{\pipebox}[4]{%
  \node[stg, fill=#4] (cell-#2-#1) at (#1,-#2) {#3};%
}
\newcommand{\pipebubble}[2]{%
  \node[bubble] (cell-#2-#1) at (#1,-#2) {\phantom{IF}};%
}

\newcommand{\plabel}[2]{\node[anchor=east,font=\scriptsize] at (0.4,-#1) {#2};}


\usepackage[framemethod=tikz]{mdframed}
\usepackage{algorithm2e}
\usepackage{listings}
\usepackage{matlab-prettifier}

% =========================
% Geometry
% =========================
\geometry{
  a4paper,
  total={160mm,240mm},
  left=30mm,
  top=25mm,
}

% =========================
% Caption setup
% =========================
\usepackage{caption}
\usepackage{subcaption}

\renewcommand{\figurename}{Figure}
\renewcommand{\tablename}{Table}

\captionsetup[figure]{labelfont={small,bf},textfont={small,it},
                      aboveskip=4pt,belowskip=2pt}
\captionsetup[table]{labelfont={small,bf},textfont={small,it},
                     aboveskip=6pt,belowskip=2pt}

% =========================
% Hyperref (đặt gần cuối)
% =========================
\usepackage[unicode]{hyperref}
\hypersetup{
  colorlinks=true,
  linkcolor=black,
  citecolor=black,
  urlcolor=blue
}

% =========================
% Header / footer
% =========================
\usepackage{fancyhdr}
\setlength{\headheight}{40pt}
\pagestyle{fancy}

\fancyhead{} % clear
\fancyfoot{} % clear

\fancyhead[L]{%
  \begin{tabular}{rl}
    \begin{picture}(25,15)(0,0)
      \put(0,-8){\includegraphics[width=8mm,height=8mm]{hcmut.png}}
    \end{picture}&
    \begin{tabular}{l}
      \textbf{\ttfamily Ho Chi Minh City University of Technology}\\
      \textbf{\ttfamily Faculty of Electrical and Electronic Engineering}
    \end{tabular}
  \end{tabular}
}

\fancyhead[R]{}

\fancyfoot[L]{\scriptsize \ttfamily Milestone 3 Report}
\fancyfoot[R]{\scriptsize \ttfamily Page \thepage/\pageref{LastPage}}

\renewcommand{\headrulewidth}{0.3pt}
\renewcommand{\footrulewidth}{0.3pt}

% =========================
% Section depth & subsubsub
% =========================
\setcounter{secnumdepth}{4}
\setcounter{tocdepth}{3}

\makeatletter
\newcounter{subsubsubsection}[subsubsection]
\renewcommand\thesubsubsubsection{\thesubsubsection.\@alph\c@subsubsubsection}
\newcommand\subsubsubsection{\@startsection{subsubsubsection}{4}{\z@}%
  {-3.25ex\@plus -1ex \@minus -.2ex}%
  {1.5ex \@plus .2ex}%
  {\normalfont\normalsize\bfseries}}
\newcommand*\l@subsubsubsection{\@dottedtocline{3}{10.0em}{4.1em}}
\newcommand*{\subsubsubsectionmark}[1]{}
\makeatother

% =========================
% Figure numbering by section
% =========================
\usepackage{chngcntr}
\counterwithin{figure}{section}
\counterwithin{table}{section}

% =========================
% Theorem environments
% =========================
\newtheorem{theorem}{{\bf Định lý}}
\newtheorem{property}{{\bf Tính chất}}
\newtheorem{proposition}{{\bf Mệnh đề}}
\newtheorem{corollary}[proposition]{{\bf Hệ quả}}
\newtheorem{lemma}[proposition]{{\bf Bổ đề}}
\theoremstyle{definition}
\newtheorem{exer}{Bài toán}

% =========================
% Listings
% =========================
\lstset{
  language=R,
  basicstyle=\footnotesize\sffamily,
  commentstyle=\ttfamily\color{black},
  numbers=left,
  numberstyle=\ttfamily\color{black}\footnotesize,
  stepnumber=1,
  numbersep=5pt,
  backgroundcolor=\color{Dandelion},
  showspaces=false,
  showstringspaces=false,
  showtabs=false,
  frame=single,
  tabsize=2,
  captionpos=b,
  breaklines=true,
  breakatwhitespace=false,
  title=\lstname,
  escapeinside={},
  keywordstyle={},
  morekeywords={}
}

\everymath{\color{black}}
\sloppy

\begin{document}

% =========================
% Title page 1
% =========================
\begin{titlepage}
  \begin{tikzpicture}[remember picture,overlay,inner sep=0,outer sep=0]
    \draw[blue!70!black,line width=4pt]
      ([xshift=-1.5cm,yshift=-2cm]current page.north east) coordinate (A) --
      ([xshift=2.4cm,yshift=-2cm]current page.north west) coordinate (B) --
      ([xshift=2.4cm,yshift=2cm]current page.south west) coordinate (C) --
      ([xshift=-1.5cm,yshift=2cm]current page.south east) coordinate (D) -- cycle;

    \draw ([yshift=0.5cm,xshift=-0.5cm]A) --
          ([yshift=0.5cm,xshift=0.5cm]B) --
          ([yshift=-0.5cm,xshift=0.5cm]B) --
          ([yshift=-0.5cm,xshift=-0.5cm]B) --
          ([yshift=0.5cm,xshift=-0.5cm]C) --
          ([yshift=0.5cm,xshift=0.5cm]C) --
          ([yshift=-0.5cm,xshift=0.5cm]C) --
          ([yshift=-0.5cm,xshift=-0.5cm]D) --
          ([yshift=0.5cm,xshift=-0.5cm]D) --
          ([yshift=0.5cm,xshift=0.5cm]D) --
          ([yshift=-0.5cm,xshift=0.5cm]A) --
          ([yshift=-0.5cm,xshift=-0.5cm]A) --
          ([yshift=0.5cm,xshift=-0.5cm]A);

    \draw ([yshift=-0.3cm,xshift=0.3cm]A) --
          ([yshift=-0.3cm,xshift=-0.3cm]B) --
          ([yshift=0.3cm,xshift=-0.3cm]B) --
          ([yshift=0.3cm,xshift=0.3cm]B) --
          ([yshift=-0.3cm,xshift=0.3cm]C) --
          ([yshift=-0.3cm,xshift=-0.3cm]C) --
          ([yshift=0.3cm,xshift=-0.3cm]C) --
          ([yshift=0.3cm,xshift=0.3cm]D) --
          ([yshift=-0.3cm,xshift=0.3cm]D) --
          ([yshift=-0.3cm,xshift=-0.3cm]D) --
          ([yshift=0.3cm,xshift=-0.3cm]A) --
          ([yshift=0.3cm,xshift=0.3cm]A) --
          ([yshift=-0.3cm,xshift=0.3cm]A);
  \end{tikzpicture}

  \begin{center}
    \textbf{\large VIETNAM NATIONAL UNIVERSITY, HO CHI MINH CITY}\\
    \textbf{\large HO CHI MINH CITY UNIVERSITY OF TECHNOLOGY}\\
    \textbf{\large Faculty of Electrical and Electronic Engineering}\\[0.7cm]

    \includegraphics[scale=1]{hcmut.png}\\[0.7cm]

    \begin{tabular}{c}
      \textbf{{\Large MILESTONE 3 REPORT}}\\[0.3cm]
      \textbf{{\Large COMPUTER ARCHITECTURE}}\\[0.3cm]
      \hline\\[-0.2cm]
      \textbf{\textit{\Large TOPIC}}\\[0.2cm]
      {\Large DESIGN OF PIPELINED RISC-V PROCESSORS}\\[0.2cm]
      \hline
    \end{tabular}
  \end{center}

  \vspace{1cm}

  \begin{table}[H]
  \centering
  \begin{tabular}{rl}
    Instructional Lecturer: & \textbf{TRAN HOANG LINH} \\
    \\[-0.3cm]
    Class:  & \textbf{L01} \\
    \\[-0.3cm]
    Group:  & \textbf{23} \\
    \\[-0.3cm]
    Academic Year: & \textbf{2025--2026} \\
    \\[-0.3cm]
    Project Source Code: & \href{https://drive.google.com/drive/folders/1HE0_QGGTYdU_0wZzMaSOYfAEO60toDNy?usp=drive_link}{\textbf{Google Drive}} \\
  \end{tabular}
\end{table}


  \vspace{1cm}
  \begin{center}
    {\footnotesize Ho Chi Minh City, January 2026}
  \end{center}
\end{titlepage}

% =========================
% Title page 2 (members)
% =========================
\begin{titlepage}
  \begin{center}
    \textbf{\large VIETNAM NATIONAL UNIVERSITY, HO CHI MINH CITY}\\
    \textbf{\large HO CHI MINH CITY UNIVERSITY OF TECHNOLOGY}\\
    \textbf{\large Faculty of Electrical and Electronic Engineering}\\[0.7cm]

    \includegraphics[scale=1]{hcmut.png}\\[0.7cm]

    \begin{tabular}{c}
      \textbf{{\Large MILESTONE 3 REPORT}}\\[0.3cm]
      \textbf{{\Large COMPUTER ARCHITECTURE}}\\[0.3cm]
      \hline\\[-0.2cm]
      \textbf{\textit{\Large TOPIC}}\\[0.2cm]
      {\Large DESIGN OF PIPELINED RISC-V PROCESSORS}\\[0.2cm]
      \hline
    \end{tabular}\\[0.8cm]

    \textbf{\large Group's members table}\\[0.3cm]

    \begin{tabular}{|c|l|c|c|}
      \hline
      & \hspace{1cm}\textbf{Name} & \textbf{Student ID} & \textbf{Contribution} \\ \hline
      1 & Nguyen Thanh Phong & 2312626 & 100\% \\ \hline
      2 & Nguyen Hoang Tuan  & 2313746 & 100\% \\ \hline
      3 & Vu Tien Tuan       & 2313774 & 0\%   \\ \hline
    \end{tabular}

    \vfill
    {\footnotesize Ho Chi Minh City, January 2026}
  \end{center}
\end{titlepage}

% =========================
% TOC
% =========================
\tableofcontents
\newpage

% =========================================================
\section{Introduction}
Milestone 3 upgrades our RV32I processor from the single-cycle design (Milestone 2) to a classic
5-stage pipelined microarchitecture consisting of \textbf{IF, ID, EX, MEM, WB}.
The main objectives are:
\begin{itemize}
  \item Implement a functional 5-stage pipeline and pass the provided ISA tests.
  \item Handle pipeline hazards by using \textbf{stall/flush} control.
  \item Evaluate and compare two baseline configurations:
  \begin{itemize}
    \item \textbf{Non-forwarding}: resolve data hazards mainly by stalling until write-back.
    \item \textbf{Forwarding}: add forwarding paths to reduce stall cycles.
  \end{itemize}
\end{itemize}

In this baseline implementation, we do \textbf{not} include a branch predictor. Instead, we use the standard
baseline policy of \textbf{predict NOT-TAKEN} (PC increments by 4) and apply \textbf{flush} when the branch/jump
is resolved as taken. Therefore, the branch misprediction rate is expected to be similar between
non-forwarding and forwarding configurations; forwarding mainly improves performance by reducing data-hazard stalls.

% =========================================================
\section{System Overview}

\subsection{Pipeline Stages}
\begin{itemize}
  \item \textbf{IF (Instruction Fetch)}: update PC, fetch instruction, compute PC+4.
  \item \textbf{ID (Instruction Decode)}: decode, read register file, generate immediate.
  \item \textbf{EX (Execute)}: ALU operations, address generation, branch/jump decision and target calculation.
  \item \textbf{MEM (Memory)}: data memory access for loads/stores (and MMIO if enabled).
  \item \textbf{WB (Writeback)}: select write-back value and write into the register file.
\end{itemize}

\subsection{Top-Level Datapath}
Figure~\ref{fig:pipeline_top} illustrates the overall pipelined datapath, including pipeline registers,
hazard control signals, and the optional forwarding unit.

\begin{figure}[H]
  \centering
  % Put this image in the same folder as main.tex or adjust path accordingly.
  \includegraphics[width=0.95\linewidth]{mile3.png}
  \caption{Top-level 5-stage pipelined RV32I processor (hazard unit + optional forwarding).}
  \label{fig:pipeline_top}
\end{figure}

\subsection{Control Flow (Predict NOT-TAKEN Baseline)}
Our baseline control policy is \textbf{predict NOT-TAKEN}:
\begin{itemize}
  \item The fetch stage always continues with PC+4 by default.
  \item Branch/jump decision is resolved in the EX stage.
  \item If the control transfer is taken, wrong-path younger instructions are invalid and must be \textbf{flushed}
        (converted into NOP-like behavior by clearing control signals).
\end{itemize}

\textbf{Debug/monitor signals (per project interface):}
\begin{itemize}
  \item \texttt{o\_ctrl}: asserted when a control-transfer instruction (branch/jump) is present in the tracked stage.
  \item \texttt{o\_mispred}: asserted when a flush occurs due to a taken control transfer under NOT-TAKEN baseline.
\end{itemize}

% =========================================================
\section{Design and Implementation}

\subsection{Reused Modules From Milestone 2}
The following blocks are reused with minimal changes:
\begin{itemize}
  \item \textbf{ALU}, \textbf{Immediate Generator}, \textbf{Register File}, and \textbf{Main Control Decoder}.
  \item \textbf{Load/Store Unit}.
\end{itemize}
The major Milestone 3 additions are pipeline registers, stage-wise control propagation, and hazard handling logic.

% -----------------------------
\subsection{Pipeline Registers and Control Propagation}
We implement four pipeline registers: \textbf{IF/ID}, \textbf{ID/EX}, \textbf{EX/MEM}, \textbf{MEM/WB}.
Each pipeline register stores both datapath signals and a bundle of control signals required by downstream stages.

\textbf{General rules:}
\begin{itemize}
  \item \textbf{Reset}: initialize stage registers into a safe NOP-like state.
  \item \textbf{Stall}: hold current contents to delay younger instructions (PC and IF/ID typically stall together).
  \item \textbf{Flush}: clear control signals so the flushed instruction produces no architectural side effects.
\end{itemize}

% -----------------------------
\subsection{Hazard Handling Overview}
In a 5-stage pipeline, hazards are categorized into:
\begin{itemize}
  \item \textbf{Data hazards (RAW)}: an instruction needs a register value that is still being produced by an earlier instruction.
  \item \textbf{Control hazards}: the next PC depends on a branch/jump decision resolved later in the pipeline.
\end{itemize}

This report focuses on data-hazard mechanisms for two configurations:
\begin{itemize}
  \item Non-forwarding: stall on most RAW hazards until producer writes back.
  \item Forwarding: forward ALU/WB results when possible and stall only in unavoidable cases (e.g., load-use).
\end{itemize}

% =========================================================
\subsection{Non-forwarding Configuration}

\subsubsection{Hazard Detection Unit Behavior}
In the non-forwarding design, the pipeline stalls when the ID-stage instruction depends on a result that is still
in-flight in EX or MEM stages. The hazard detection unit compares the source registers of the ID instruction
(\texttt{rs1\_ID}, \texttt{rs2\_ID}) against the destination registers of the instructions in EX and MEM
(\texttt{rd\_EX}, \texttt{rd\_MEM}), and uses \texttt{regWEn} to filter only instructions that will write back.

To avoid false hazards, the input \texttt{uses\_rs2} is used:
\begin{itemize}
  \item If the current ID instruction does not consume \texttt{rs2} (e.g., I-type ALU ops, loads),
        the hazard check against \texttt{rs2\_ID} is disabled.
\end{itemize}

When a hazard is detected, the unit forces:
\begin{itemize}
  \item \textbf{PC stall}: \texttt{pc\_en = 0} (PC is held).
  \item \textbf{IF/ID stall}: \texttt{IF\_ID\_en = 0} (ID instruction is held).
  \item \textbf{Insert bubble}: \texttt{ID\_EX\_flush = 1} (clear control signals into EX so EX gets a NOP).
\end{itemize}

\subsubsection{Hazard Logic Summary}
Let \texttt{rd\_EX} and \texttt{rd\_MEM} be the destination registers of EX/MEM instructions.
A stall is asserted when:
\begin{itemize}
  \item EX produces a register write and its destination matches \texttt{rs1\_ID}, or matches \texttt{rs2\_ID} when \texttt{uses\_rs2=1}.
  \item MEM produces a register write and its destination matches \texttt{rs1\_ID}, or matches \texttt{rs2\_ID} when \texttt{uses\_rs2=1}.
\end{itemize}
Because there is no forwarding, the consumer must wait until the value is finally written back, resulting in
multiple stall cycles depending on where the producer currently is in the pipeline.

% =========================================================
\subsection{Forwarding Configuration}

\subsubsection{Forwarding Unit (EX Operand Selection)}
The forwarding unit reduces stalls by selecting the most recent value for EX operands directly from later pipeline stages.
For the instruction currently in EX:
\begin{itemize}
  \item If a matching destination register exists in MEM stage and writes back, forward from MEM (\texttt{forward=2'b10}).
  \item Otherwise, if a matching destination register exists in WB stage and writes back, forward from WB (\texttt{forward=2'b01}).
  \item Else, use the original operand read from register file (\texttt{forward=2'b00}).
\end{itemize}

We apply \textbf{priority} MEM over WB because MEM contains newer data for back-to-back dependencies.

\subsubsection{Forwarding Control Signals}
Two 2-bit control signals are generated:
\begin{itemize}
  \item \texttt{forwardA\_EX}: select source for EX operand A (rs1 path).
  \item \texttt{forwardB\_EX}: select source for EX operand B (rs2 path).
\end{itemize}
Destination register \texttt{rd=0} is ignored to preserve x0 semantics.

\subsubsection{Hazard Detection Unit Behavior}
Even with forwarding, the \textbf{load-use} hazard typically cannot be resolved in the same cycle because load data is only
available after MEM. Therefore, we stall only for the standard load-use condition:
\begin{itemize}
  \item If EX-stage instruction is a load (opcode \texttt{0000011}) and its destination \texttt{rd\_EX} matches \texttt{rs1\_ID} or \texttt{rs2\_ID},
        then stall PC and IF/ID and insert a bubble into EX.
\end{itemize}

This makes the forwarding pipeline much less stall-heavy than the non-forwarding pipeline.

\subsubsection{Hazard Logic Summary}
With forwarding enabled, most RAW hazards are resolved without stalling by selecting the newest available value
for EX operands from later stages. Forwarding decisions are based on register-number matching and write-enable signals:

\begin{itemize}
  \item \textbf{Forward from MEM (highest priority):}
  if \texttt{regWEn\_MEM=1}, \texttt{rd\_MEM$\neq$0}, and \texttt{rd\_MEM = rsX\_EX} $\Rightarrow$ select MEM result.
  \item \textbf{Forward from WB:}
  else if \texttt{regWEn\_WB=1}, \texttt{rd\_WB$\neq$0}, and \texttt{rd\_WB = rsX\_EX} $\Rightarrow$ select WB result.
  \item \textbf{No forwarding:}
  otherwise use the register-file value captured in ID/EX.
\end{itemize}

Despite forwarding, a \textbf{load-use} dependency still requires a one-cycle stall because load data becomes available
only after the MEM stage. The pipeline stalls when the EX instruction is a load and its destination \texttt{rd\_EX}
matches \texttt{rs1\_ID} or \texttt{rs2\_ID}; in that case PC and IF/ID are held, and a bubble is inserted into ID/EX.

% =========================================================
\subsection{Stall/Flush Policy Summary}
Table~\ref{tab:stall_flush_policy} summarizes the stall/flush behavior used in both configurations.

\begin{table}[H]
\centering
\small
\setlength{\tabcolsep}{5pt}
\renewcommand{\arraystretch}{1.2}

\begin{tabularx}{\linewidth}{
  >{\raggedright\arraybackslash}p{3.6cm}
  >{\centering\arraybackslash}p{1.2cm}
  >{\centering\arraybackslash}p{1.4cm}
  >{\centering\arraybackslash}p{1.5cm}
  >{\raggedright\arraybackslash}X
}
\toprule
\textbf{Scenario} & \textbf{PC enable} & \textbf{IF/ID enable} & \textbf{ID/EX flush} & \textbf{Effect} \\
\midrule
No hazard &
1 & 1 & 0 &
Normal pipeline flow. \\

Non-fwd RAW hazard (EX/MEM producer) &
0 & 0 & 1 &
Hold PC and IF/ID; insert a bubble in ID/EX until the producer value is available at write-back. \\

Fwd load-use hazard (EX is load) &
0 & 0 & 1 &
One-cycle stall; after that, forwarding resolves the dependency. \\

Taken branch/jump (predict NOT-TAKEN) &
1 & \textit{Flush} & \textit{Flush} &
Redirect PC to the target and flush younger wrong-path instructions in IF/ID and ID/EX. \\
\bottomrule
\end{tabularx}

\caption{Stall/flush policy summary for non-forwarding and forwarding configurations.}
\label{tab:stall_flush_policy}
\end{table}


% =========================================================
\section{Verification}

\subsection{Simulation Methodology}
Verification is performed in two layers:

\begin{itemize}
  \item \textbf{Directed unit testcases (Forwarding configuration):}
  We write short micro-programs to intentionally create specific data-dependency patterns so that the
  forwarding unit must select the correct source (\texttt{EX/MEM} or \texttt{MEM/WB}) and obey the
  priority rule (MEM over WB). For each testcase, we provide a pipeline timeline and a typed
  \textit{Results} log showing the expected forwarding select codes.

  \item \textbf{Course-provided ISA tests (Non-forwarding and Forwarding configurations):}
  After validating the forwarding behavior with directed testcases, we run the official ISA regression
  suite provided by the course. Correctness is determined by the testbench \texttt{PASS/FAIL} criteria.
  At the end of each run, the testbench reports total executed cycles, total executed instructions, IPC/CPI,
  and branch misprediction statistics. These ISA results are used for the final comparison between
  the non-forwarding and forwarding baselines.
\end{itemize}


\subsection{Directed Unit Testcases}
\subsubsection{Forwarding Unit Testcases (Forwarding configuration)}
To validate the forwarding unit, we create directed micro-programs that force different forwarding sources:
\textbf{EX/MEM (MEM stage)} and \textbf{MEM/WB (WB stage)}. In our design, the selection codes are:
\texttt{2'b10} = forward from MEM (\texttt{alu\_mem}), \texttt{2'b01} = forward from WB (\texttt{wb\_data}),
and \texttt{2'b00} = no forwarding (use ID/EX values). MEM has higher priority than WB.

% =========================================================
\paragraph{TC-F1: Forward from MEM (EX/MEM) for rs1}\par\noindent
\begin{lstlisting}[style=asm,caption={TC-F1 (MEM forwarding): back-to-back dependency},label={lst:tc_f1}]
add  x5, x1, x2
sub  x6, x5, x3    # needs x5 immediately -> forwardA_EX = 2'b10
\end{lstlisting}

\begin{figure}[H]
\centering
\begin{tikzpicture}[x=1.05cm,y=0.75cm]
  % cycle grid
  \foreach \c in {1,...,7} {
    \draw[gray!25] (\c,0.6) -- (\c,-1.9);
    \node[font=\scriptsize] at (\c,0.85) {\c};
  }
  % labels
  \plabel{0}{add x5, x1, x2}
  \plabel{1}{sub x6, x5, x3}

  % add row (row=0)
  \pipebox{1}{0}{IF}{white}
  \pipebox{2}{0}{ID}{IDc}
  \pipebox{3}{0}{EX}{EXc}
  \pipebox{4}{0}{MEM}{MEMc}
  \pipebox{5}{0}{WB}{WBc}

  % sub row (row=1)
  \pipebox{2}{1}{IF}{white}
  \pipebox{3}{1}{ID}{IDc}
  \pipebox{4}{1}{EX}{EXc}
  \pipebox{5}{1}{MEM}{MEMc}
  \pipebox{6}{1}{WB}{WBc}

  % forwarding arrow (MEM -> EX at cycle 4)  (edge-to-edge)
\draw[fwdarrow]
  (cell-0-4.west)
  to[out=180,in=180,looseness=1.25]
  (cell-1-4.west);

\end{tikzpicture}
\caption{TC-F1 pipeline timeline (forward from MEM to EX for rs1).}
\label{fig:tc_f1}
\end{figure}

\begin{tcolorbox}[breakable,title={Results},colback=white,colframe=black]
\texttt{3: inst\_EX=add x5,x1,x2, ForwardA: 00, ForwardB: 00, PC\_fw\_A: 0, PC\_fw\_B: 0}\\
\texttt{4: inst\_EX=sub x6,x5,x3, ForwardA: 10, ForwardB: 00, PC\_fw\_A: 0, PC\_fw\_B: 0}
\end{tcolorbox}


% =========================================================
\paragraph{TC-F2: Forward from WB (MEM/WB) for rs2}\par\noindent
\begin{lstlisting}[style=asm,caption={TC-F2 (WB forwarding): one-cycle gap dependency},label={lst:tc_f2}]
add  x5, x1, x2
addi x0, x0, 0      # nop to create 1-cycle gap
sub  x6, x3, x5     # x5 available in WB -> forwardB_EX = 2'b01
\end{lstlisting}

\begin{figure}[H]
\centering
\begin{tikzpicture}[x=1.05cm,y=0.75cm]
  \foreach \c in {1,...,8} {
    \draw[gray!25] (\c,0.6) -- (\c,-2.6);
    \node[font=\scriptsize] at (\c,0.85) {\c};
  }
  \plabel{0}{add x5, x1, x2}
  \plabel{1}{addi x0, x0, 0}
  \plabel{2}{sub x6, x3, x5}

  % add
  \pipebox{1}{0}{IF}{white}
  \pipebox{2}{0}{ID}{IDc}
  \pipebox{3}{0}{EX}{EXc}
  \pipebox{4}{0}{MEM}{MEMc}
  \pipebox{5}{0}{WB}{WBc}

  % addi (nop)
  \pipebox{2}{1}{IF}{white}
  \pipebox{3}{1}{ID}{IDc}
  \pipebox{4}{1}{EX}{EXc}
  \pipebox{5}{1}{MEM}{MEMc}
  \pipebox{6}{1}{WB}{WBc}

  % sub
  \pipebox{3}{2}{IF}{white}
  \pipebox{4}{2}{ID}{IDc}
  \pipebox{5}{2}{EX}{EXc}
  \pipebox{6}{2}{MEM}{MEMc}
  \pipebox{7}{2}{WB}{WBc}

  % forwarding arrow (WB -> EX at cycle 5)
  \draw[fwdarrow]
  (cell-0-5.west)
  to[out=180,in=180,looseness=0.7]
  (cell-2-5.west);

\end{tikzpicture}
\caption{TC-F2 pipeline timeline (forward from WB to EX for rs2).}
\label{fig:tc_f2}
\end{figure}

\begin{tcolorbox}[breakable,title={Results},colback=white,colframe=black]
\texttt{3: inst\_EX=add  x5,x1,x2,   ForwardA: 00, ForwardB: 00, PC\_fw\_A: 0, PC\_fw\_B: 0}\\
\texttt{4: inst\_EX=addi x0,x0,0,    ForwardA: 00, ForwardB: 00, PC\_fw\_A: 0, PC\_fw\_B: 0}\\
\texttt{5: inst\_EX=sub  x6,x3,x5,   ForwardA: 00, ForwardB: 01, PC\_fw\_A: 0, PC\_fw\_B: 0}
\end{tcolorbox}


% =========================================================
\paragraph{TC-F3: MEM has priority over WB (mixed sources)}\par\noindent
\begin{lstlisting}[style=asm,caption={TC-F3 (priority): MEM over WB},label={lst:tc_f3}]
add  x5, x1, x2
add  x6, x3, x4
sub  x7, x5, x6     # at EX: x6 in MEM, x5 in WB -> rs2 from MEM, rs1 from WB
\end{lstlisting}

\begin{figure}[H]
\centering
\begin{tikzpicture}[x=1.05cm,y=0.75cm]
  \foreach \c in {1,...,9} {
    \draw[gray!25] (\c,0.6) -- (\c,-2.6);
    \node[font=\scriptsize] at (\c,0.85) {\c};
  }
  \plabel{0}{add x5, x1, x2}
  \plabel{1}{add x6, x3, x4}
  \plabel{2}{sub x7, x5, x6}

  % add x5
  \pipebox{1}{0}{IF}{white}
  \pipebox{2}{0}{ID}{IDc}
  \pipebox{3}{0}{EX}{EXc}
  \pipebox{4}{0}{MEM}{MEMc}
  \pipebox{5}{0}{WB}{WBc}

  % add x6
  \pipebox{2}{1}{IF}{white}
  \pipebox{3}{1}{ID}{IDc}
  \pipebox{4}{1}{EX}{EXc}
  \pipebox{5}{1}{MEM}{MEMc}
  \pipebox{6}{1}{WB}{WBc}

  % sub x7
  \pipebox{3}{2}{IF}{white}
  \pipebox{4}{2}{ID}{IDc}
  \pipebox{5}{2}{EX}{EXc}
  \pipebox{6}{2}{MEM}{MEMc}
  \pipebox{7}{2}{WB}{WBc}


% MEM row1 -> EX row2 (cong hơn + ra trái hơn để né)
\draw[fwdarrow]
  (cell-1-5.west)
  to[out=180,in=180,looseness=0.7]
  (cell-2-5.west);

\end{tikzpicture}
\caption{TC-F3 pipeline timeline (mixed forwarding sources; MEM has higher priority than WB).}
\label{fig:tc_f3}
\end{figure}

\begin{tcolorbox}[breakable,title={Results},colback=white,colframe=black]
\texttt{3: inst\_EX=add x5,x1,x2, ForwardA: 00, ForwardB: 00, PC\_fw\_A: 0, PC\_fw\_B: 0}\\
\texttt{4: inst\_EX=add x6,x3,x4, ForwardA: 00, ForwardB: 00, PC\_fw\_A: 0, PC\_fw\_B: 0}\\
\texttt{5: inst\_EX=sub x7,x5,x6, ForwardA: 01, ForwardB: 10, PC\_fw\_A: 0, PC\_fw\_B: 0}
\end{tcolorbox}

\subsubsection{Hazard Unit Testcases (Forwarding configuration)}
With forwarding enabled, the hazard unit stalls only for the classic load-use case because load data is
available after MEM. During a stall, PC and IF/ID are held and one bubble (NOP) is injected into EX.
Bubbles are shown as gray patterned cells without text.

\paragraph{TC-H1: Load-use stall (lw $\rightarrow$ add)}\par\noindent
\begin{lstlisting}[style=asm,caption={TC-H1 (load-use): one-cycle stall then forward from WB},label={lst:tc_h1}]
lw   x5, 0(x1)
add  x6, x5, x2     # uses loaded x5 -> must stall 1 cycle
\end{lstlisting}

\begin{figure}[H]
\centering
\begin{tikzpicture}[x=1.05cm,y=0.75cm]
  \foreach \c in {1,...,8} {
    \draw[gray!25] (\c,0.6) -- (\c,-3.3);
    \node[font=\scriptsize] at (\c,0.85) {\c};
  }

  \plabel{0}{lw x5, 0(x1)}
  \plabel{1}{bubble (NOP)}
  \plabel{2}{add x6, x5, x2}

  % lw row0
  \pipebox{1}{0}{IF}{white}
  \pipebox{2}{0}{ID}{IDc}
  \pipebox{3}{0}{EX}{EXc}
  \pipebox{4}{0}{MEM}{MEMc}
  \pipebox{5}{0}{WB}{WBc}

  % bubble row1 (injected into EX at cycle 4), no text
  \pipebubble{4}{1}{}
  \pipebubble{5}{1}{}
  \pipebubble{6}{1}{}

  % add row2 (ID held for 1 extra cycle)
  \pipebox{2}{2}{IF}{white}
  \pipebox{3}{2}{ID}{IDc}
  \pipebox{4}{2}{ID}{IDc}   % stall: hold in ID
  \pipebox{5}{2}{EX}{EXc}
  \pipebox{6}{2}{MEM}{MEMc}
  \pipebox{7}{2}{WB}{WBc}

  % after stall, forward from WB (lw) -> EX (add) at cycle 5
  \draw[fwdarrow]
    (cell-0-5.west)
    to[out=180,in=180,looseness=0.7]
    (cell-2-5.west);
\end{tikzpicture}
\caption{TC-H1 pipeline timeline (one-cycle load-use stall; bubble injected into EX).}
\label{fig:tc_h1}
\end{figure}

\begin{tcolorbox}[breakable,title={Results},colback=white,colframe=black]
\texttt{3: HZD=1 (pc\_en=0, IF\_ID\_en=0, ID\_EX\_flush=1)}\\
\texttt{4: inst\_EX=bubble (NOP)}\\
\texttt{5: inst\_EX=add x6,x5,x2, ForwardA: 01, ForwardB: 00, PC\_fw\_A: 0, PC\_fw\_B: 0}
\end{tcolorbox}

\paragraph{TC-H2: Load-use stall (lw $\rightarrow$ sw uses rs2)}\par\noindent
\begin{lstlisting}[style=asm,caption={TC-H2 (load-use): lw then sw depends on rs2},label={lst:tc_h2}]
lw   x5, 0(x1)
sw   x5, 0(x2)      # store-data uses x5 (rs2) -> must stall 1 cycle
\end{lstlisting}

\begin{figure}[H]
\centering
\begin{tikzpicture}[x=1.05cm,y=0.75cm]
  \foreach \c in {1,...,8} {
    \draw[gray!25] (\c,0.6) -- (\c,-3.3);
    \node[font=\scriptsize] at (\c,0.85) {\c};
  }

  \plabel{0}{lw x5, 0(x1)}
  \plabel{1}{bubble (NOP)}
  \plabel{2}{sw x5, 0(x2)}

  % lw row0
  \pipebox{1}{0}{IF}{white}
  \pipebox{2}{0}{ID}{IDc}
  \pipebox{3}{0}{EX}{EXc}
  \pipebox{4}{0}{MEM}{MEMc}
  \pipebox{5}{0}{WB}{WBc}

  % bubble row1
  \pipebubble{4}{1}{}
  \pipebubble{5}{1}{}
  \pipebubble{6}{1}{}

  % sw row2 (ID held 1 cycle)
  \pipebox{2}{2}{IF}{white}
  \pipebox{3}{2}{ID}{IDc}
  \pipebox{4}{2}{ID}{IDc}
  \pipebox{5}{2}{EX}{EXc}
  \pipebox{6}{2}{MEM}{MEMc}
  \pipebox{7}{2}{WB}{WBc}

  % forward store-data from WB (lw) -> EX (sw) at cycle 5 (rs2 path)
  \draw[fwdarrow]
    (cell-0-5.west)
    to[out=180,in=180,looseness=0.7]
    (cell-2-5.west);
\end{tikzpicture}
\caption{TC-H2 pipeline timeline (lw$\rightarrow$sw load-use stall; bubble injected into EX).}
\label{fig:tc_h2}
\end{figure}

\begin{tcolorbox}[breakable,title={Results},colback=white,colframe=black]
\texttt{3: HZD=1 (pc\_en=0, IF\_ID\_en=0, ID\_EX\_flush=1)}\\
\texttt{4: inst\_EX=bubble (NOP)}\\
\texttt{5: inst\_EX=sw x5,0(x2), ForwardA: 00, ForwardB: 01, PC\_fw\_A: 0, PC\_fw\_B: 0}
\end{tcolorbox}


\subsection{ISA Test Results}
Figures~\ref{fig:isa_nonfwd} and~\ref{fig:isa_fwd} show the result summaries for non-forwarding and forwarding configurations.

\begin{figure}[H]
  \centering
  \includegraphics[width=0.90\linewidth]{non_fwd.png}
  \caption{ISA test summary (Non-forwarding configuration).}
  \label{fig:isa_nonfwd}
\end{figure}

\begin{figure}[H]
  \centering
  \includegraphics[width=0.90\linewidth]{fwd.png}
  \caption{ISA test summary (Forwarding configuration).}
  \label{fig:isa_fwd}
\end{figure}

% =========================================================
\section{Performance and Results}

\subsection{Metrics}
We report three key performance metrics:
\begin{itemize}
  \item \textbf{IPC (Instructions Per Cycle)}: higher IPC indicates fewer stalls and better pipeline utilization.
  \item \textbf{CPI (Cycles Per Instruction)}: reciprocal of IPC; lower CPI is better.
  \item \textbf{Branch Misprediction Rate}: with predict NOT-TAKEN baseline, taken control transfers cause flushes and increase mispredictions.
\end{itemize}

\subsection{ISA Test Performance (Non-forwarding vs Forwarding)}
We report ISA test results only.
Table~\ref{tab:isa_perf} summarizes the measured results.

\begin{table}[H]
\centering
\begin{tabular}{lrr}
\toprule
\textbf{Metric} & \textbf{Non-forwarding} & \textbf{Forwarding} \\
\midrule
Total Clock Cycles Executed & 11132 & 7841 \\
Total Instructions Executed & 4831 & 4831 \\
IPC & 0.43 & 0.62 \\
CPI & 2.30 & 1.62 \\
Total Branch Instructions & 1604 & 1604 \\
Total Branch Mispredictions & 1453 & 1453 \\
Branch Misprediction Rate & 90.59\% & 90.59\% \\
\bottomrule
\end{tabular}
\caption{ISA test performance comparison between non-forwarding and forwarding configurations.}
\label{tab:isa_perf}
\end{table}

\subsection{Discussion}
Forwarding provides a clear speedup on the ISA tests. Compared to the non-forwarding design,
the forwarding design reduces total cycles from 11132 to 7841, i.e., 3291 fewer cycles (about 29.6\% reduction).
This corresponds to an overall speedup of approximately 1.42$\times$.

The IPC improves from 0.43 to 0.62 because forwarding eliminates many stall cycles caused by data hazards.
Consequently, CPI drops from 2.30 to 1.62.

The branch misprediction rate remains the same at 90.59\% for both configurations because both designs use the same
\textbf{predict NOT-TAKEN} baseline policy. Forwarding improves data-hazard handling but does not change the control-hazard behavior.

\subsection{Quartus Timing and Resource Utilization}
For Quartus post-fit evaluation, both the instruction memory (IMEM) and data memory (DMEM) are configured to
\textbf{2K 32-bit words} each. Timing results (Fmax, setup slack, hold slack) are taken from TimeQuest reports,
while logic/register/memory utilization is taken from the Fitter summary.

\begin{table}[H]
\centering
\begin{tabular}{lcc}
\toprule
\textbf{Metric} & \textbf{Non-forwarding} & \textbf{Forwarding} \\
\midrule
Fmax (MHz) & 52.41 & 51.98 \\
Worst Setup Slack (ns) & 0.921 & 0.762 \\
Worst Hold Slack (ns) & 0.215 & 0.236 \\
Total Logic Elements (count or \%) & 9,219 / 14,448 (64\%) & 9,141 / 14,448 (63\%) \\
Total Registers (count or \%) & 1,858 & 1,743 \\
Total Memory Bits (count or \%) & 65,536 / 239,616 (27\%) & 65,536 / 239,616 (27\%) \\
\bottomrule
\end{tabular}
\caption{Quartus post-fit timing and resource summary.}
\label{tab:quartus_summary}
\end{table}



% =========================================================
\section{Conclusion}
We successfully implemented a 5-stage pipelined RV32I processor and verified correctness using the course ISA tests.
Two baseline configurations were evaluated:
\begin{itemize}
  \item \textbf{Non-forwarding}: correct execution is achieved by stalling on RAW hazards until results are written back.
  \item \textbf{Forwarding}: forwarding paths significantly reduce stall cycles and improve pipeline utilization.
\end{itemize}

On the ISA tests, forwarding reduces total cycles by about 29.6\% and achieves roughly 1.42$\times$ speedup,
increasing IPC from 0.43 to 0.62. The branch misprediction rate remains unchanged because both configurations
use the same predict NOT-TAKEN control policy.

% =========================================================
\section{Future Work}
Possible improvements include:
\begin{itemize}
  \item Implement a branch predictor to reduce control-hazard penalty and lower misprediction rate.
  \item Optimize critical paths to improve Fmax and timing slack in Quartus.
  \item Add additional benchmarks beyond ISA tests for broader performance evaluation.
\end{itemize}

% =========================================================
\begin{thebibliography}{9}
\bibitem{patterson}
D. A. Patterson and J. L. Hennessy,
\textit{Computer Organization and Design: RISC-V Edition},
Morgan Kaufmann, 2017.
\end{thebibliography}

\end{document}
